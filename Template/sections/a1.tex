\newpage
\subsection{Vorverarbeitung}
Um die folgende Verarbeitung der Bilder zu vereinfachen und robuster zu gestalten, sollen Sie den Videostream mit einem Preprocessing vorverarbeiten. 
Binden Sie daf�r die Datei \textit{./CV-App/algorithms/invis\_cloak.py} in den Algorithmus ein, wie in der Einleitung beschrieben.
Die folgenden Aufgabenstellungen sind in den daf�r vorgesehenen Funktionen zu bearbeiten.
%Erstellen Sie daf�r einen neuen Algorithmus, wie in der Einleitung beschrieben. 

\subsubsection{Rauschreduktion}
Jeder Farbwert eines Pixels $I_k(x, y) \in \{0, \ldots, 255 \}$ mit $k\in \{R, G, B\}$ wird auf dem Kamerasensor durch einen elektrischen Halbleiter physikalisch gemessen. Je nach Sensorqualit�t und Lichtbedingungen wirkt dabei ein unterschiedlich ausgepr�gtes Rauschen auf die Farbwerte ein, sodass der zur Verf�gung stehende Farbwert als Summe

\begin{equation}
I_k(x, y) = I^*_k(x, y) + r(x, y)   
\end{equation}

aus realem Farbwert $I^*_k(x, y) $ und statistischem Rauschen $r(x,y)$ modelliert werden kann. Das Rauschen $r$ kann als normalverteilt um den Mittelwert $0$ angenommen werden. Unter den Annahmen, dass die Kamera statisch montiert ist und in der aufgenommenen Szene keine Ver�nderung passiert,  kann der Zusammenhang

\begin{equation}
	\overline{I}_{k,t}(x, y) =   \lim_{N\rightarrow \infty} \frac{1}{N + 1} \sum_{n=0}^N I^*_{k,t-n}(x, y) + r_{t-n}(x, y)  \stackrel{!}{=} I^*_{k,t}  
\end{equation}

f�r die Mittelwertbildung �ber lange Zeitr�ume formuliert werden. Dabei beschreibt $t$ den Zeitpunkt, zu dem der entsprechende Wert gemessen wurde.

Um die Bildqualit�t zu erh�hen, soll der Einfluss von $r$ reduziert werden. Es soll daf�r angenommen werden, dass die Kamera statisch ist und kaum Bewegung in zwei aufeinander folgenden Bildern vorhanden ist. 
Implementieren Sie die Mittelwertbildung mit einer variablem Bildreihe $N$ (default: $N=1$) und geben Sie das Bild aus. \\
Um zu pr�fen wie das Bild auf Pixelebene arbeitet, kann die Variable \textit{plotNoise} in der Funktion \textit{process()} auf \textit{True} gesetzt werden.
Es werden zwei zus�tzliche Plots ausgegeben, in der ein Bildausschnitt des Zentrums vor- und nach der Rauschunterdr�ckung vergr��ert dargestellt werden.


\paragraph*{Aufgabe 1}
Geben Sie Ihren Code an und beschreiben Sie ihn. Geben Sie nur relevante Code Bereiche an!
 \lstset{caption={Vorverarbeitung, Aufgabe 1}}
\begin{lstlisting}
# Your code!
\end{lstlisting} 

\paragraph*{Aufgabe 2}
Nennen Sie Vor und Nachteile, wenn $N$ vergr��ert werden w�rde. Sollte $N$ in dieser Anwendung vergr��ert werden?

\paragraph*{Aufgabe 3}
Beschreiben Sie eine weitere Methode zur Rauschreduktion. Diskutieren Sie dabei Vor- oder Nachteile!

\subsubsection{Histogramm Spreizung}
Pixel k�nnen in unserer Anwendung Werte von $I_k(x,y) \in \{ 0, \ldots , 255 \}$ annehmen. Dieser Wertebereich wird nicht zwangsl�ufig ausgenutzt. Um das zu �ndern, soll eine Histogramm Spreizung auf den Helligkeitsinformationen der Pixel durchgef�hrt werden.

Implementieren Sie zus�tzlich zur Rauschreduktion eine Histogramm Spreizung, indem sie (1) das Rausch-reduzierte Eingangsbild in den HSV-Farbbereich transformieren und (2) die Rechenvorschrift~\ref{equ:histogramm-equalization} auf den V-Kanal anwenden. Transformieren Sie das Bild dann (3) wieder in den RGB Farbraum.

\begin{equation}
\label{equ:histogramm-equalization}
I_V^{\textnormal{new}}(x,y) = \frac{I_{V}(x,y) - \min I_{V}}{\max I_{V} - \min I_{V}} \cdot 255
\end{equation}

\textbf{Hinweis:} Nutzen Sie die Befehle \textit{cv2.cvtColor(img, cv2.COLOR\_BGR2HSV)} beziehungsweise \textit{cv2.cvtColor(img, cv2.COLOR\_HSV2BGR)}.


\paragraph*{Aufgabe 4}
Geben Sie Ihren Code an und beschreiben Sie ihn. Geben Sie nur relevante Code Bereiche an!
\lstset{caption={Vorverarbeitung, Aufgabe 4}}
\begin{lstlisting}
# Your code!
\end{lstlisting} 

\paragraph*{Aufgabe 5}
Warum ist es sinnvoll, den gesamten Wertebereich f�r die Darstellung von Videos in Multimedia-Anwendungen auszunutzen?