\newpage
\subsection{Segmentierung und Bildmodifizierung}
In diesem Arbeitspaket werden Sie auf Grundlage der vorherigen Analysen eine Segmentierung des magischen Umhangs realisieren. Anschlie�end werden Sie den segmentierten Bereich \glqq verschwinden\grqq\ lassen, indem sie ein statisches Bild des Hintergrunds auf diese Fl�chen einf�gen.


\subsubsection{Statisches Schwellwertverfahren}
Implementieren Sie die von Ihnen gefundene Regel nach Gleichung~\ref{equ:segrule}, um eine Bin�rmaske zu erhalten. Sie k�nnen die Randbedingungen wie im folgenden Code-Schnipsel \ref{lst:conditions} implementieren.

\begin{lstlisting}[caption={Benutzung von Randbedingungen mit \textit{numpy}}, label={lst:conditions}]
import cv2 
import numpy as np

channel1 = 0
lower_bound1, upper_bound1 = 15, 100
is_condition_1_true = (lower_bound1 < img[:, :, channel1]) * \
	(img[:, :, channel1] < upper_bound1)
channel2 = 1
lower_bound2, upper_bound2 = 65, 172
is_condition_2_true = (lower_bound2 < img[:, :, channel2]) * \
(img[:, :, channel2] < upper_bound2)

binary_mask = is_condition_1_true * is_condition_2_true
\end{lstlisting} 

Geben Sie die gefundene Bin�rmaske als Ausgangsbild auf dem Bildschirm aus. Sollten die gefundenen Wertebereich zu keinen sinnvollen Segmentierungen f�hren, d�rfen Sie Gleichung~\ref{equ:segrule} selbstverst�ndlich anpassen!

Implementieren Sie ebenfalls eine Mausklick-Funktion, mit der Sie das aktuelle Bild und die dazugeh�rige Bin�rmaske abspeichern k�nnen. F�r das Abspeichern von Bildern k�nnen Sie die Funktion \textit{cv2.imwrite(img, "path\_to\_store.png")} verwenden.

\paragraph*{Aufgabe 1}
Geben Sie Ihren Code an und beschreiben Sie ihn. Geben Sie nur relevante Code Bereiche an! Geben Sie ebenfalls das aufgenommene Bild sowie die dazugeh�rige Bin�rmaske an.

\lstset{caption={Segmentierung und Bildmodifizierung, Aufgabe 1}}
\begin{lstlisting}
# Your code!
\end{lstlisting} 

\subsubsection{Bin�rmaske}
Die in der vorherigen Aufgabe erhaltene Bin�rmaske ist ggf.\ ungeeignet f�r eine zufriedenstellende Segmentierung. Sie sollen die Maske nun optimieren. Wenden Sie daf�r das \textit{Opening} und \textit{Closing} auf die Bin�rmaske an.  Nutzen Sie die Funktionen \textit{cv2.erode(img, kernel)} und \textit{cv2.dilate(img, kernel)}. 

W�hlen Sie zum Schluss die gr��te zusammenh�nge Region segmentierter Pixel aus, und l�schen alle anderen Segmente. Folgender Code-Schnipsel~\ref{lst:contours} soll als Hilfestellung dienen. Recherchieren Sie ggf.\ im Internet.  

\lstset{caption={Konturfindung}}
\lstset{label={lst:contours}}
\begin{lstlisting}
(cnts, _) = cv2.findContours(
		binary_mask, cv2.RETR_LIST, cv2.CHAIN_APPROX_SIMPLE)
c = max(cnts, key = cv2.contourArea)
img = cv2.drawContours(img, [c], -1, color=255, -1)
\end{lstlisting} 

\paragraph*{Aufgabe 2}
Geben Sie Ihren Code an und beschreiben Sie ihn. Geben Sie nur relevante Code Bereiche an!
\lstset{caption={Segmentierung und Bildmodifizierung, Aufgabe 2}}
\begin{lstlisting}
# Your code!
\end{lstlisting} 

\paragraph*{Aufgabe 3}
Welche Probleme oder Fehler k�nnen in der Bin�rmaske vorkommen, die mit den Ma�nahmen beseitigt werden sollen?

\subsubsection{Bildmodifizierung}
Nach dem Fertigstellen der vorherigen Aufgabenstellungen sollten Sie nun eine Bin�rmaske erhalten, welche den \glqq magischen Umhang\grqq\ segmentiert. Die letzte Aufgabe befasst sich mit der Bildmodifizierung, welche den Eindruck verschwindender Objekte vermittelt.

Sie sollen nun folgende Funktionen implementieren: 

Erstellen Sie eine Member-Variable (z.B. \textit{self.variable}) in die Algorithmus Funktion \textit{\_\_init\_\_()}. Initiieren Sie die Variable mit dem Wert \textit{None}.
Modifizieren Sie den Algorithmus, sodass Sie mit einem Mausklick ein Bild in die Variable speichern k�nnen. Dieses Bild wird als Hintergrund definiert. Mausklick-Funktionen aus vorherigen Aufgaben k�nnen �berschrieben werden! 

Solange kein Bild in der Variable gespeichert ist, soll das Eingangsbild direkt wieder ausgegeben werden. Sobald ein Hintergrund vorhanden ist soll folgendes passieren: Modifizieren Sie das Bild, indem Sie das Ausgangsbild aus dem derzeitigen Kamera-Stream und dem Hintergrund zusammenf�gen. Die durch die Bin�rmaske segmentierte Fl�che soll aus dem Hintergrund entnommen werden, die unsegmentierte Fl�che aus dem derzeitigen Videostream.

\textbf{Hinweis:} Verlassen Sie das Sichtfeld der Kamera, w�hrend Sie die Hintergrund Aufnahme aufnehmen. 

\paragraph*{Aufgabe 4}
Geben Sie Ihren Code an und beschreiben Sie ihn. Geben Sie nur relevante Code Bereiche an! 
\lstset{caption={Segmentierung und Bildmodifizierung, Aufgabe 4}}
\begin{lstlisting}
# Your code!
\end{lstlisting} 

\paragraph*{Aufgabe 5}
Geben Sie ein Bild (z.B. Screenshot) an, in dem die Funktion Ihres \glqq magischen Umhangs\grqq\  gezeigt wird!